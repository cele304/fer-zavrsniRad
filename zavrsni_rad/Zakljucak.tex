U ovom radu detaljno je proučena teorija i analiza kombinatornih igara. Teorija pruža alate i koncepte koji omogućuju modeliranje i analizu strateške interakcije u igrama. Kroz primjenu Grundyjeve funkcije, koja se temelji na teoriji grafova, možemo odrediti pobjedničke strategije u kombinatornim igrama.

Naglasak u izradi ovog rada je na progresivnim kombinatornim igrama, s posebnim fokusom na analizi niza modifikacija igre Nim. Primjena analitičke tehnike iz teorije grafova omogućila je dublje razumijevanje strukture i dinamike igre i identifikaciju optimalnih strategija za postizanje pobjede.

Uz teorijsku analizu, razvijena je i interaktivna web aplikacija koja implementira jednu verziju igre Nim. Ova aplikacija omogućuje korisnicima da igraju igru i istraže različite scenarije i strategije. 

S druge strane, kroz matematičke formalizme i analitičke tehnike, možemo bolje razumjeti kompleksne igre i identificirati optimalne strategije. Ovi rezultati imaju široku primjenu u područjima kao što su računalna znanost, matematika, inženjerstvo i društvene znanosti.

U zaključku, ovaj rad pruža sveobuhvatnu analizu kombinatornih igara. Njegovi rezultati i primjena Grundyjeve funkcije ukazuju na mogućnosti optimizacije strategija u igri Nim i sličnim situacijama.