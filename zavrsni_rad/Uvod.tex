Teorija grafova vrlo je važna grana matematike koja se primjenjuje u mnogim poljima znanosti, kao što su računalna znanost, inženjerstvo, ali i u društvenim znanostima. Koncept grafa je apstraktni model koji se koristi za rješavanje različitih problema u ljudskom djelovanju. S druge strane, kombinatorne igre predstavljaju matematičke modele za stratešku interakciju između racionalnih dionika.

Ovaj rad proučava progresivne kombinatorne igre i predstavlja apstraktnu analizu njihovih pobjedničkih strategija koristeći teoriju grafova. Konkretno, u fokusu je Grundyjeva funkcija, koja se koristi za određivanje ishoda kombinatornih igara.

Analiza pobjedničkih strategija ključna je jer pomaže u razumijevanju procesa donošenja odluka i pruža uvid u međusobnu stratešku komunikaciju. Korištenje teorije grafova u ovoj analizi omogućuje nam modeliranje složenih sustava i njihovu strukturiranu i detaljnu analizu.

Nadalje, primijenjena je razrađena analitička tehnika iz teorije grafova za analizu i odabir optimalnih pobjedničkih strategija za niz modifikacija igre Nim. Nim je klasična kombinatorna igra koja je opsežno proučavana u literaturi i ima brojne primjene u računalnim znanostima i inženjerstvu.

Kako bismo analizu kombinatornih igara učinili što pristupačnijom i kao rezultat cjelokupne teorijske analize, izrađena je interaktivna web aplikacija koja implementira jednu verziju igre Nim. To omogućuje korisnicima da eksperimentiraju s igrom i bolje usavrše razumijevanje koncepata igre o kojima se govori u ovom radu.

Sveukupno, cilj je ovog rada pružiti sveobuhvatnu analizu kombinatornih igara i njihovih pobjedničkih strategija korištenjem teorije grafova.
